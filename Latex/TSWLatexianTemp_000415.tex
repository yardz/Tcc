% ------------------------------------------------------------------------
% ------------------------------------------------------------------------
% abnTeX2: Modelo de Trabalho Academico (tese de doutorado, dissertacao de
% mestrado e trabalhos monograficos em geral) em conformidade com 
% ABNT NBR 14724:2011: Informacao e documentacao - Trabalhos academicos -
% Apresentacao
% ------------------------------------------------------------------------
% ------------------------------------------------------------------------

\documentclass[
	% -- opções da classe memoir --
	12pt,				% tamanho da fonte
	openright,			% capítulos começam em pág ímpar (insere página vazia caso preciso)
	twoside,			% para impressão em verso e anverso. Oposto a oneside
	a4paper,			% tamanho do papel. 
	Times,
	% -- opções da classe abntex2 --
	%chapter=TITLE,		% títulos de capítulos convertidos em letras maiúsculas
	%section=TITLE,		% títulos de seções convertidos em letras maiúsculas
	%subsection=TITLE,	% títulos de subseções convertidos em letras maiúsculas
	%subsubsection=TITLE,% títulos de subsubseções convertidos em letras maiúsculas
	% -- opções do pacote babel --
%	english,			% idioma adicional para hifenização
	brazil,				% o último idioma é o principal do documento
	]{abntex2}


%=================================================================================================
% IMPORTAÇÃO DE PACOTES
%=================================================================================================

%=======================================================
% Pacotes fundamentais 
%=======================================================
\usepackage{cmap}				% Mapear caracteres especiais no PDF
\usepackage{lmodern}			% Usa a fonte Latin Modern			
\usepackage[T1]{fontenc}		% Selecao de codigos de fonte.
\usepackage[utf8]{inputenc}		% Codificacao do documento (conversão automática dos acentos)
\usepackage{lastpage}			% Usado pela Ficha catalográfica
\usepackage{indentfirst}		% Indenta o primeiro parágrafo de cada seção.
\usepackage{color}				% Controle das cores
\usepackage{graphicx}			% Inclusão de gráficos

%=======================================================
% Pacotes para URL
%=======================================================
%\usepackage{url}            % deve ser carregado para tratar url de forma correta e usado pelo abntex2cite
%\usepackage{breakurl}    % trata de forma correta a quebra de url entre linhas
% ---

%=======================================================
% Pacote da PUC Minas
%=======================================================
\usepackage[combrasao]{abntex2-pucminas}

%=======================================================
% Pacote de Citações
%=======================================================
\usepackage[brazilian,hyperpageref]{backref}	 % Paginas com as citações na bibl
\usepackage[alf]{abntex2cite}	% Citações padrão ABNT



%=================================================================================================
% CONFIGURAÇÕES DE PACOTES
%=================================================================================================

%=======================================================
% Configurações do pacote backref
%=======================================================
% Usado sem a opção hyperpageref de backref
\renewcommand{\backrefpagesname}{Citado na(s) página(s):~}
% Texto padrão antes do número das páginas
\renewcommand{\backref}{}


\renewcommand*{\backrefalt}[4]{
%	\ifcase #1 
%		Nenhuma citação no texto.%
%	\or
%		Citado na página #2.%
%	\else
%		Citado #1 vezes nas páginas #2.%
%	\fi
	}%
% ---


%=================================================================================================
% Informações de dados para CAPA e FOLHA DE ROSTO
%=================================================================================================
\titulo{Referencial Teórico}
\autor{Bruno Motta Azevedo do Nascimento}
\local{Belo Horizonte, MG}
\data{2014}
\orientador{João Caram}
\instituicao{
  Pontifícia Universidade Católica de Minas Gerais
  \par
  Bacharelado em Sistemas de Informação
 }
\tipotrabalho{Trabalho de Conclusão de Curso}
% O preambulo deve conter o tipo do trabalho, o objetivo, 
% o nome da instituição e a área de concentração 
\preambulo{Trabalho de conclusão de curso do aluno Bruno Motta Azevedo do Nascimento. Este é um trabalho que visa o desenvolvimento de uma ferramenta web para gestão de conteúdo multimidia.}


%=================================================================================================
% Configurações de aparência do PDF final
%=================================================================================================

%=======================================================
% Alterando Cores
%=======================================================
% alterando o aspecto da cor azul
\definecolor{blue}{RGB}{41,5,195}

%=======================================================
% informações do PDF
%=======================================================
\makeatletter
\hypersetup{
     	pagebackref=true,
		pdftitle={\@title}, 
		pdfauthor={\@author},
    	pdfsubject={\imprimirpreambulo},
	    pdfcreator={LaTeX with abnTeX2},
		pdfkeywords={wordpress}{web}{gestão de conteúdo}{CMS}{trabalho acadêmico}, 
		colorlinks=true,       		% false: boxed links; true: colored links
    	linkcolor=blue,          	% color of internal links
    	citecolor=blue,        		% color of links to bibliography
    	filecolor=magenta,      		% color of file links
		urlcolor=blue,
		bookmarksdepth=4
}
\makeatother

%=======================================================
% Espaçamentos entre linhas e parágrafos 
%=======================================================
% O tamanho do parágrafo é dado por:
\setlength{\parindent}{1.3cm}
% Controle do espaçamento entre um parágrafo e outro:
\setlength{\parskip}{0.2cm}  % tente também \onelineskip

%=======================================================
% Compila o indice
%=======================================================
\makeindex

%=================================================================================================
% Início do documento
%=================================================================================================
\begin{document}

%=======================================================
% Retira espaço extra obsoleto entre as frases.
%=======================================================
%\frenchspacing 

%=================================================================================================
% ELEMENTOS PRÉ-TEXTUAIS
%=================================================================================================
 %\pretextual

%=======================================================
% Capa
%=======================================================
\imprimircapa
% ---

%=======================================================
% Folha de rosto
%=======================================================
% (o * indica que haverá a ficha bibliográfica)
\imprimirfolhaderosto*

%=======================================================
% Inserir a ficha bibliografica
%=======================================================
%\begin{fichacatalografica}
%	\includepdf{fig_ficha_catalografica.pdf}
%\end{fichacatalografica}
%\begin{fichacatalografica}
%	\vspace*{\fill}					% Posição vertical
%	\hrule							% Linha horizontal
%	\begin{center}					% Minipage Centralizado
%	\begin{minipage}[c]{12.5cm}		% Largura
%	
%	\imprimirautor
%	
%	\hspace{0.5cm} \imprimirtitulo  / \imprimirautor. --
%	\imprimirlocal, \imprimirdata-
%	
%	\hspace{0.5cm} \pageref{LastPage} p. : il. (algumas color.) ; 30 cm.\\
%	
%	\hspace{0.5cm} \imprimirorientadorRotulo~\imprimirorientador\\
%	
%	\hspace{0.5cm}
%	\parbox[t]{\textwidth}{\imprimirtipotrabalho~--~\imprimirinstituicao,
%	\imprimirdata.}\\
%	
%	\hspace{0.5cm}
%		1. Wordpress.
%		2. Gestão de conteúdo.
%		3. Internet.
%		I. Orientador:  João Caram.
%		II. Universidade Pontifícia Universidade Católica de Minas Gerais.
%		III. Faculdade de Sistemas de Informação.
%		IV. Referencial Teórico.		
%	\hspace{8.75cm} CDU 02:141:005.7\\
%	
%	\end{minipage}
%	\end{center}
%	\hrule
%\end{fichacatalografica}




%=======================================================
% Inserir errata
%=======================================================
%\begin{errata}
%Elemento opcional da \citeonline[4.2.1.2]{NBR14724:2011}. Exemplo:
%
%\vspace{\onelineskip}
%
%FERRIGNO, C. R. A. \textbf{Tratamento de neoplasias ósseas apendiculares com
%reimplantação de enxerto ósseo autólogo autoclavado associado ao plasma
%rico em plaquetas}: estudo crítico na cirurgia de preservação de membro em
%cães. 2011. 128 f. Tese (Livre-Docência) - Faculdade de Medicina Veterinária e
%Zootecnia, Universidade de São Paulo, São Paulo, 2011.
%
%
%\begin{table}[htb]
%\center
%\footnotesize
%\begin{tabular}{|p{1.4cm}|p{1cm}|p{3cm}|p{3cm}|}
%  \hline
%   \textbf{Folha} & \textbf{Linha}  & \textbf{Onde se lê}  & \textbf{Leia-se}  \\
%    \hline
%    1 & 10 & auto-conclavo & autoconclavo\\
%   \hline
%\end{tabular}
%\end{table}
%
%\end{errata}




%=======================================================
% Inserir folha de aprovação
%=======================================================

% Isto é um exemplo de Folha de aprovação, elemento obrigatório da NBR
% 14724/2011 (seção 4.2.1.3). Você pode utilizar este modelo até a aprovação
% do trabalho. Após isso, substitua todo o conteúdo deste arquivo por uma
% imagem da página assinada pela banca com o comando abaixo:
%
% \includepdf{folhadeaprovacao_final.pdf}
%

%Folha de aprovação
%\begin{folhadeaprovacao}
%
%  \begin{center}
%    {\ABNTEXchapterfont\large\imprimirautor}
%
%    \vspace*{\fill}\vspace*{\fill}
%    {\ABNTEXchapterfont\bfseries\Large\imprimirtitulo}
%    \vspace*{\fill}
%    
%    \hspace{.45\textwidth}
%    \begin{minipage}{.5\textwidth}
%        \imprimirpreambulo
%    \end{minipage}%
%    \vspace*{\fill}
%   \end{center}
%    
%   Trabalho aprovado. \imprimirlocal, 24 de novembro de 2012:
%
%   \assinatura{\textbf{\imprimirorientador} \\ Orientador} 
%   \assinatura{\textbf{Professor} \\ Convidado 1}
%   \assinatura{\textbf{Professor} \\ Convidado 2}
%   %\assinatura{\textbf{Professor} \\ Convidado 3}
%   %\assinatura{\textbf{Professor} \\ Convidado 4}
%      
%   \begin{center}
%    \vspace*{0.5cm}
%    {\large\imprimirlocal}
%    \par
%    {\large\imprimirdata}
%    \vspace*{1cm}
%  \end{center}
%  
%\end{folhadeaprovacao}





%=======================================================
% Dedicatória
%=======================================================
%\begin{dedicatoria}
%   \vspace*{\fill}
%   \centering
%   \noindent
%   \textit{ Este trabalho é dedicado às crianças adultas que,\\
%   quando pequenas, sonharam em se tornar cientistas.} \vspace*{\fill}
%\end{dedicatoria}



%=======================================================
% Agradecimentos
%=======================================================
%\begin{agradecimentos}
%Os agradecimentos principais são direcionados à Gerald Weber, Miguel Frasson,
%Leslie H. Watter, Bruno Parente Lima, Flávio de Vasconcellos Corrêa, Otavio Real
%Salvador, Renato Machnievscz\footnote{Os nomes dos integrantes do primeiro
%projeto abn\TeX\ foram extraídos de
%\url{http://codigolivre.org.br/projects/abntex/}} e todos aqueles que
%contribuíram para que a produção de trabalhos acadêmicos conforme
%as normas ABNT com \LaTeX\ fosse possível.
%
%Agradecimentos especiais são direcionados ao Centro de Pesquisa em Arquitetura
%da Informação\footnote{\url{http://www.cpai.unb.br/}} da Universidade de
%Brasília (CPAI), ao grupo de usuários
%\emph{latex-br}\footnote{\url{http://groups.google.com/group/latex-br}} e aos
%novos voluntários do grupo
%\emph{\abnTeX}\footnote{\url{http://groups.google.com/group/abntex2} e
%\url{http://abntex2.googlecode.com/}}~que contribuíram e que ainda
%contribuirão para a evolução do \abnTeX.
%
%\end{agradecimentos}
% ---

% ---
% Epígrafe
% ---
%\begin{epigrafe}
%    \vspace*{\fill}
%	\begin{flushright}
%		\textit{Tudo o que temos de decidir é o que fazer com o tempo que nos é dado. - Gandalf}
%	\end{flushright}
%\end{epigrafe}
% ---
















%=================================================================================================
% RESUMOS
%=================================================================================================

%=======================================================
% resumo em português
%=======================================================
\setlength{\absparsep}{18pt} % ajusta o espaçamento dos parágrafos do resumo
\begin{resumo}
 Este trabalho tem como objetivo servir de referencial teórico para a monografia do aluno Bruno Motta. Ele serve para contextualizar o ambiente do problema a ser resolvido, assim como introduzir a ferramenta que será usada como base para a resolução deste problema.
 \textbf{Palavras-chaves}: Wordpress, Internet, Gestão de Conteúdo.
\end{resumo}

%=======================================================
% resumo em inglês
%=======================================================
%\begin{resumo}[Abstract]
% \begin{otherlanguage*}{english}
%   This is the english abstract.
%
%   \vspace{\onelineskip}
% 
%   \noindent 
%   \textbf{Key-words}: latex. abntex. text editoration.
% \end{otherlanguage*}
%\end{resumo}
% ---


%=================================================================================================
% Listas e Indices
%=================================================================================================

%=======================================================
% inserir lista de ilustrações
%=======================================================
%\pdfbookmark[0]{\listfigurename}{lof}
%\listoffigures*
%\cleardoublepage

%=======================================================
% inserir lista de tabelas
%=======================================================
%\pdfbookmark[0]{\listtablename}{lot}
%\listoftables*
%\cleardoublepage

%=======================================================
% inserir lista de abreviaturas e siglas
%=======================================================
%\begin{siglas}
%  \item[Fig.] Area of the $i^{th}$ component
%  \item[456] Isto é um número
%  \item[123] Isto é outro número
%  \item[lauro cesar] este é o meu nome
%\end{siglas}

%=======================================================
% inserir lista de símbolos
%=======================================================
%\begin{simbolos}
%  \item[$ \Gamma $] Letra grega Gama
%  \item[$ \Lambda $] Lambda
%  \item[$ \zeta $] Letra grega minúscula zeta
%  \item[$ \in $] Pertence
%\end{simbolos}

%=======================================================
% inserir o sumario
%=======================================================
\pdfbookmark[0]{\contentsname}{toc}
\tableofcontents*
\cleardoublepage


%=================================================================================================
% ELEMENTOS TEXTUAIS
%=================================================================================================
\textual

%=======================================================
% Introdução
%=======================================================
%\chapter*[Introdução]{Introdução}
%\addcontentsline{toc}{chapter}{Introdução}



%=======================================================
% TEXTO
%=======================================================

\chapter{Internet}
\section{Primórdios}

A internet pode ser considerada um produto vindo da guerra. Na busca desenfreada por uma vantagem durante a guerra fria os EUA queriam uma forma de ser ver a frente da URSS, e com isso buscavam sempre o desenvolvimento de novas tecnologias e inovações. \cite{historia-internt}
\par

Foi J.C.R. Licklider, do Instituto Tecnológico de Massachussets (MIT), que no ano de 1962 difundiu a idéia de  “rede galáctica”, que seriam todos os computadores da terra conectados com uma única forma de compartilhamento, idéia audaciosa, porem nos dias de hoje não é tão improvável assim. \cite{historia-internt}
\par

\subsection{ARPANET}

A ARPANET é considerada por muitos a percursora da internet. Ela foi desenvolvida durante o período histórico conhecido como "Guerra Fria", pelos EUA que temendo um ataque soviético pudesse vir a perder tanto a comunicação quanto informações. \cite{ARPANET}
\par

A primeira mensagem foi enviada no ano de 1969, a ARPANET enviou a sua primeira mensagem com a palavra "login", que chegou incompleta por alguns erros, ao qual foram gastos alguns anos para a correção dos mesmos. \cite{historia-internt}
\par

\subsection{Década de 70}

Muitas coisas foram feitas na década de 1970. Os cientistas Vinton Cerf e sua equipe, fizeram um experimento que era a conexão de três redes distintas, tal projeto era chamado de "interneting", termo que foi sendo vastamente usado e abreviado até receber o nome de Internet. \cite{historia-internt}
\par

Ainda no ano de 1970 Vinton Cerf, desenvolveu o protocolo TCP/IP que visava transmitir as mensagens sem erro, da origem ao destino. Porem tal protocolo só foi oficializado como único no ano de  1983.\cite{historia-internt}
\par

Ainda na mesma década, a primeira mensagem contendo um emoticon foi enviada, por Kevin MacKenzie que usou um símbolo para representar uma ironia. Em 1971, Bob Thomas, criou o que hoje conhecemos como vírus, que não fazia nada além de incomodar o usuário. \cite{historia-internt}
\par

Já no ano de 1973, a ARPANET foi ligada a Europa e a primeira conexão intercontinental foi estabelecida. Em 1977, nos EUA, já havia um número bem maior que os quatro primeiros servidores da ARPANET. Em 1979 Tom Truscott e Jim Ellis, interligaram computadores em uma rede de notícias divididos por categorias de interesse, o que fez com a rede não fosse somente interessante para maios científicos, mas sim para diversas pessoas.  \cite{historia-internt}
\par


\subsection{MILNET, ARPANET, Internet}

Em 1983 com o crescimento da ARPANET e com o declínio da "Guerra Fria" a ARPANET, perdeu uma parte de seu caráter militar, e por este motivo, ela foi dividida em duas novas redes a MILNET que era militar e a ARPANET, que se tornou de uso civil. Em 1989 o segmento civil, foi desativado, porem serviu de para diversas redes interonectadas, o que hoje chamamos de Internet. \cite{historia-internt, web}
\par

\chapter{Evolução da Web}

\section{Web}

Em 1989, Tim Berners-Lee, criou a World Wide Web o que chamamos de WEB. Esse projeto tinha como objetivo ligar as universidades para compartilhar e usar mutuamente os trabalhos acadêmicos. Esse mesmo cientista criou o código HTML e o protocolo HTTP.  \cite{web}
\par

\subsection{Guerra dos navegadores}
Em 1993, Marc Andreessen e Rob McCool inventaram o primeiro navegador chamado de Mosaic. Por dois anos, o Mosaic foi o principal meio para pessoas fora das universidades explorem a internet. O Objetivo dos criadores do programa era tornar o uso da internet algo prático para outras pessoas. Alguns anos depois  diversos navegadores foram surgindo, revolucionando o software com recursos novos e possibilidade de personalização. \cite{web}
\par

Em 1995 surge o Netscape, que foi o primeiro navegador comercial da internet, algum tempo depois a Microsoft lançou o seu navegador o Internet Explorer que passou a ser usado pela grande maioria dos usuários da rede. \cite{web}
\par

\section{Web 2.0}

Como foi notado o objetivo da internet e da web sofreu muitas mudanças, passou de uma forma de manter e compartilhar informações militares, a uma rede que pode potencialmente conectar todos os computadores do mundo e partilhar informações, fazer amigos, descobrir coisas novas dentre diversas outras possibilidades. Neste momento pode-se perceber que a antiga Web, já havia extrapolado seu escopo inicial, e por isso havia se tornado algo novo, algo que chamamos de WEB 2.0. \cite{ARPANET, historia-internt, web, web2.0}
\par

A Web 2.0 não está também vinculada apenas a documentos científicos por uma rede. Ela mudou os objetivos com o passar do tempo. Na atualidade compartilhar arquivos acadêmicos entre universidades não pode mais ser chamado de prioridade. \cite{web2.0}
\par

Hoje o objetivo da chamada Web 2.0 vai muito além disso, temos hoje uma rede onde cada usuário tem o potencial de gerar informações, de participar para construir algo tanto de forma direta e cociente quanto de forma indireta e inconsciente. O Google se faz de exemplo da forma como vários usuários de forma inconsciente melhoram os serviços oferecidos pela empresa, com ferramentas como PageRank que é uma lista de popularidade muito bem trabalhada.\cite{google, web2.0}
\par



\section{Inteligência coletiva}
A inteligência coletiva no ambiente web, ser vista como uma forma. \cite{inteligencia-coletiva, inteligencia-coletiva-web}
\par

A idéia de Inteligência coletiva, é de fazer o uso de várias pessoas, para contribuir com a melhoria de um determinado serviço. O Google, novamente citado, faz o uso desse poder da rede em seu mecanismo de busca, através do que ele chama de PageRank. \cite{google}
\par



\section{Classificação}

Desenvolver TAGS, “folksonomia”, Resumo, Introdução, Cronograma, Qual é o problema, solução proposta. \par


%
%\chapter{Modelos de serviço}
%Agora iremos explicar alguns dos tipos de serviços que as aplicações devem ofertar a seus usuários finais. Esses são serviços específicos de aplicações em nuvem, ao qual vende-se algo que não é essencialmente um serviço como se este o fosse\cite{web2.0}.
%\par
%
%\section{Infraestrutura como um serviço (IaaS)}
%Esse modelo de serviço tem como objetivo fazer com que a gerência dos recursos de infra-estrutura dos servidores. Ele fornece aos contratantes a possibilidade de gerenciar os recursos computacionais necessários. Este serviço está ligado diretamente com a capacidade de virtualização de hardware\cite{saas, nuvem, nuvem2}.
%\par
%OBS: já comentado. Categorizar, exemplificar, diferenciar, etc.
%\par
%
%\section{Plataforma como um serviço (PaaS)}
%Esta camada intermediária da nuvem, ela permite que os desenvolvedores possam criar e implementar as aplicações sem o conhecimento do hardware que irá executar a tarefa. Neste nível não há controle dos recursos computacionais, mas sim de algumas aplicações que eventualmente podem ser usadas\cite{saas, nuvem, nuvem2}.
%\par
%OBS: já comentado. Categorizar, exemplificar, diferenciar, etc.
%\par
%
%\section{Software como um serviço (SaaS)}
%As aplicações devem estar acessíveis aos clientes, a idéia de software como serviço vem da capacidade de vender um acesso ao software e não a venda do próprio software em si. Essa forma de serviço é amplamente difundida na internet em várias formas, tais como os serviços de busca e servidores de email. Neste nível a responsabilidade é disponibilizar uma aplicação completa ao usuário com um ambiente computacional completamente desconhecido para o usuário\cite{saas, nuvem, nuvem2}.
%\par
%OBS: já comentado. Categorizar, exemplificar, diferenciar, etc.
%\par
%



%=======================================================
% Finaliza a parte no bookmark do PDF, para que se inicie o bookmark na raiz
%=======================================================
\bookmarksetup{startatroot}% 
% ---

%=======================================================
% Conclusão
%=======================================================
%\chapter*[Conclusão]{Conclusão}
%\addcontentsline{toc}{chapter}{Conclusão}


%=================================================================================================
% ELEMENTOS PÓS-TEXTUAIS
%=================================================================================================
\postextual


%=======================================================
% Referências bibliográficas
%=======================================================
\bibliography{Referencial}

%=======================================================
% Glossário
%=======================================================
% Consulte o manual da classe abntex2 para orientações sobre o glossário.
%\glossary

%=================================================================================================
% Apêndices
%=================================================================================================

%=======================================================
% Inicia os apêndices
%=======================================================
%\begin{apendicesenv}
% Imprime uma página indicando o início dos apêndices
%\partapendices
%
%\chapter{Quisque libero justo}
%
%\lipsum[50]
%
% ----------------------------------------------------------
%\chapter{Nullam elementum urna vel imperdiet sodales elit ipsum pharetra ligula
%ac pretium ante justo a nulla curabitur tristique arcu eu metus}
% ----------------------------------------------------------
%\lipsum[55-57]
%
%\end{apendicesenv}
% ---


%=================================================================================================
% Anexos
%=================================================================================================

%=======================================================
% Inicia os anexos
%=======================================================
%\begin{anexosenv}
%
% Imprime uma página indicando o início dos anexos
%\partanexos
%
% ---
%\chapter{Morbi ultrices rutrum lorem.}
% ---
%\lipsum[30]
%
% ---
%\chapter{Cras non urna sed feugiat cum sociis natoque penatibus et magnis dis
%parturient montes nascetur ridiculus mus}
% ---
%
%\lipsum[31]
%
% ---
%\chapter{Fusce facilisis lacinia dui}
% ---
%
%\lipsum[32]
%
%\end{anexosenv}


%=======================================================
% INDICE REMISSIVO
%=======================================================

\printindex

\end{document}







