% ------------------------------------------------------------------------
% ------------------------------------------------------------------------
% abnTeX2: Modelo de Trabalho Academico (tese de doutorado, dissertacao de
% mestrado e trabalhos monograficos em geral) em conformidade com 
% ABNT NBR 14724:2011: Informacao e documentacao - Trabalhos academicos -
% Apresentacao
% ------------------------------------------------------------------------
% ------------------------------------------------------------------------

\documentclass[
	% -- opções da classe memoir --
	12pt,				% tamanho da fonte
	openright,			% capítulos começam em pág ímpar (insere página vazia caso preciso)
	twoside,			% para impressão em verso e anverso. Oposto a oneside
	a4paper,			% tamanho do papel. 
	Times,
	% -- opções da classe abntex2 --
	%chapter=TITLE,		% títulos de capítulos convertidos em letras maiúsculas
	%section=TITLE,		% títulos de seções convertidos em letras maiúsculas
	%subsection=TITLE,	% títulos de subseções convertidos em letras maiúsculas
	%subsubsection=TITLE,% títulos de subsubseções convertidos em letras maiúsculas
	% -- opções do pacote babel --
%	english,			% idioma adicional para hifenização
	brazil,				% o último idioma é o principal do documento
	]{abntex2}


%=================================================================================================
% IMPORTAÇÃO DE PACOTES
%=================================================================================================

%=======================================================
% Pacotes fundamentais 
%=======================================================
%\usepackage[brazil]{babel}
%\usepackage{cmap}				% Mapear caracteres especiais no PDF
%\usepackage{lmodern}			% Usa a fonte Latin Modern			
%\usepackage[utf8]{inputenc}		% Codificacao do documento (conversão automática dos acentos)
%\usepackage[T1]{fontenc}			% Selecao de codigos de fonte.
%\usepackage{lastpage}			% Usado pela Ficha catalográfica
%\usepackage{indentfirst}			% Indenta o primeiro parágrafo de cada seção.
%\usepackage{color}				% Controle das cores
%\usepackage{graphicx}			% Inclusão de gráficos
\usepackage{tabularx}			%pacote de tabelas
%=======================================================
% Pacotes para URL
%=======================================================
%\usepackage{url}            % deve ser carregado para tratar url de forma correta e usado pelo abntex2cite
%\usepackage{breakurl}    % trata de forma correta a quebra de url entre linhas


%=================================================================================================
% CONFIGURAÇÕES DE PACOTES
%=================================================================================================


%=======================================================
% Pacote da PUC Minas
%=======================================================
% O Pacote da PUC deixa tudo nos conformes para a monografia
\usepackage[combrasao]{abntex2-pucminas}


%=================================================================================================
% Informações de dados para CAPA e FOLHA DE ROSTO
%=================================================================================================
\titulo{Web 2.0}
\subtitulo{Solucionando problemas de gestão de conteúdo, fazendo uso da força da inteligência coletiva.}
\autor{Bruno Motta Azevedo do Nascimento}
\local{Belo Horizonte, MG}
\data{2014}
\orientador{João Caram}
\instituicao{Pontifícia Universidade Católica de Minas Gerais}
\departamento{Bacharelado em Sistemas de Informação}
\tipotrabalho{Trabalho de Conclusão de Curso}
\preambulo{Monografia apresentada ao Curso de Sistemas de Informação da \imprimirinstituicao, como requisito parcial para obtenção do título de Bacharel em Sistemas de Informação.}


%=================================================================================================
% Configurações de aparência do PDF final
%=================================================================================================

%=======================================================
% informações do PDF
%=======================================================
\makeatletter
\hypersetup{
     	pagebackref=true,
		pdftitle={\@title}, 
		pdfauthor={\@author},
    	pdfsubject={\imprimirpreambulo},
	    pdfcreator={LaTeX with abnTeX2},
		pdfkeywords={web}{web2.0}{gestão de conteúdo}{inteligência coletiva}, 
		colorlinks=true,       		% false: boxed links; true: colored links
    	linkcolor=blue,          			% color of internal links
    	citecolor=blue,        			% color of links to bibliography
    	filecolor=magenta,      		% color of file links
		urlcolor=blue,
		bookmarksdepth=4
}
\makeatother

%=======================================================
% Espaçamentos entre linhas e parágrafos 
%=======================================================
% O tamanho do parágrafo é dado por:
%\setlength{\parindent}{1.3cm}
% Controle do espaçamento entre um parágrafo e outro:
%\setlength{\parskip}{0.2cm}  % tente também \onelineskip

%=======================================================
% Compila o indice
%=======================================================
\makeindex

%=================================================================================================
% Início do documento
%=================================================================================================
\begin{document}
%=======================================================
% Retira espaço extra obsoleto entre as frases.
%=======================================================
%\frenchspacing 

%=================================================================================================
% ELEMENTOS PRÉ-TEXTUAIS
%=================================================================================================
 %\pretextual

%=======================================================
% Capa
%=======================================================
\imprimircapa
% ---

%=======================================================
% Folha de rosto
%=======================================================
% (o * indica que haverá a ficha bibliográfica)
\imprimirfolhaderosto*

%=======================================================
% Inserir a ficha bibliografica
%=======================================================
%\begin{fichacatalografica}
%	\includepdf{fig_ficha_catalografica.pdf}
%\end{fichacatalografica}
%\begin{fichacatalografica}
%	\vspace*{\fill}					% Posição vertical
%	\hrule							% Linha horizontal
%	\begin{center}					% Minipage Centralizado
%	\begin{minipage}[c]{12.5cm}		% Largura
%	
%	\imprimirautor
%	
%	\hspace{0.5cm} \imprimirtitulo  / \imprimirautor. --
%	\imprimirlocal, \imprimirdata-
%	
%	\hspace{0.5cm} \pageref{LastPage} p. : il. (algumas color.) ; 30 cm.\\
%	
%	\hspace{0.5cm} \imprimirorientadorRotulo~\imprimirorientador\\
%	
%	\hspace{0.5cm}
%	\parbox[t]{\textwidth}{\imprimirtipotrabalho~--~\imprimirinstituicao,
%	\imprimirdata.}\\
%	
%	\hspace{0.5cm}
%		1. Wordpress.
%		2. Gestão de conteúdo.
%		3. Internet.
%		I. Orientador:  João Caram.
%		II. Universidade Pontifícia Universidade Católica de Minas Gerais.
%		III. Faculdade de Sistemas de Informação.
%		IV. Referencial Teórico.		
%	\hspace{8.75cm} CDU 02:141:005.7\\
%	
%	\end{minipage}
%	\end{center}
%	\hrule
%\end{fichacatalografica}




%=======================================================
% Inserir errata
%=======================================================
%\begin{errata}
%Elemento opcional da \citeonline[4.2.1.2]{NBR14724:2011}. Exemplo:
%
%\vspace{\onelineskip}
%
%FERRIGNO, C. R. A. \textbf{Tratamento de neoplasias ósseas apendiculares com
%reimplantação de enxerto ósseo autólogo autoclavado associado ao plasma
%rico em plaquetas}: estudo crítico na cirurgia de preservação de membro em
%cães. 2011. 128 f. Tese (Livre-Docência) - Faculdade de Medicina Veterinária e
%Zootecnia, Universidade de São Paulo, São Paulo, 2011.
%
%
%\begin{table}[htb]
%\center
%\footnotesize
%\begin{tabular}{|p{1.4cm}|p{1cm}|p{3cm}|p{3cm}|}
%  \hline
%   \textbf{Folha} & \textbf{Linha}  & \textbf{Onde se lê}  & \textbf{Leia-se}  \\
%    \hline
%    1 & 10 & auto-conclavo & autoconclavo\\
%   \hline
%\end{tabular}
%\end{table}
%
%\end{errata}




%=======================================================
% Inserir folha de aprovação
%=======================================================

%
% \includepdf{folhadeaprovacao_final.pdf}
%

%Folha de aprovação
%\begin{folhadeaprovacao}
%
%  \begin{center}
%    {\ABNTEXchapterfont\large\imprimirautor}
%
%    \vspace*{\fill}\vspace*{\fill}
%    {\ABNTEXchapterfont\bfseries\Large\imprimirtitulo}
%    \vspace*{\fill}
%    
%    \hspace{.45\textwidth}
%    \begin{minipage}{.5\textwidth}
%        \imprimirpreambulo
%    \end{minipage}%
%    \vspace*{\fill}
%   \end{center}
%    
%   Trabalho aprovado. \imprimirlocal, 24 de novembro de 2012:
%
%   \assinatura{\textbf{\imprimirorientador} \\ Orientador} 
%   \assinatura{\textbf{Professor} \\ Convidado 1}
%   \assinatura{\textbf{Professor} \\ Convidado 2}
%   %\assinatura{\textbf{Professor} \\ Convidado 3}
%   %\assinatura{\textbf{Professor} \\ Convidado 4}
%      
%   \begin{center}
%    \vspace*{0.5cm}
%    {\large\imprimirlocal}
%    \par
%    {\large\imprimirdata}
%    \vspace*{1cm}
%  \end{center}
%  
%\end{folhadeaprovacao}





%=======================================================
% Dedicatória
%=======================================================
%\begin{dedicatoria}
%   \vspace*{\fill}
%   \centering
%   \noindent
%   \textit{ Este trabalho é dedicado às crianças adultas que,\\
%   quando pequenas, sonharam em se tornar cientistas.} \vspace*{\fill}
%\end{dedicatoria}



%=======================================================
% Agradecimentos
%=======================================================
%\begin{agradecimentos}
%Os agradecimentos principais são direcionados à Gerald Weber, Miguel Frasson,
%Leslie H. Watter, Bruno Parente Lima, Flávio de Vasconcellos Corrêa, Otavio Real
%Salvador, Renato Machnievscz\footnote{Os nomes dos integrantes do primeiro
%projeto abn\TeX\ foram extraídos de
%\url{http://codigolivre.org.br/projects/abntex/}} e todos aqueles que
%contribuíram para que a produção de trabalhos acadêmicos conforme
%as normas ABNT com \LaTeX\ fosse possível.
%
%Agradecimentos especiais são direcionados ao Centro de Pesquisa em Arquitetura
%da Informação\footnote{\url{http://www.cpai.unb.br/}} da Universidade de
%Brasília (CPAI), ao grupo de usuários
%\emph{latex-br}\footnote{\url{http://groups.google.com/group/latex-br}} e aos
%novos voluntários do grupo
%\emph{\abnTeX}\footnote{\url{http://groups.google.com/group/abntex2} e
%\url{http://abntex2.googlecode.com/}}~que contribuíram e que ainda
%contribuirão para a evolução do \abnTeX.
%
%\end{agradecimentos}
% ---

% ---
% Epígrafe
% ---
%\begin{epigrafe}
%    \vspace*{\fill}
%	\begin{flushright}
%		\textit{Tudo o que temos de decidir é o que fazer com o tempo que nos é dado. - Gandalf}
%	\end{flushright}
%\end{epigrafe}
% ---
















%=================================================================================================
% RESUMOS
%=================================================================================================

%=======================================================
% resumo em português
%=======================================================
\setlength{\absparsep}{18pt} % ajusta o espaçamento dos parágrafos do resumo
\begin{resumo}
	Atualmente a internet é um meio de comunicação altamente difundido. Com a velocidade e abundância de informação na Web, um grupo de pessoas começou a se interessar mais por animes, que são animações típicas da cultura japonesa. Esses animes, atualmente podem ser encontrados em abundância na Web com diversas temáticas e para todas as idades. O grande volume de animes diferentes, atrapalha aos interessados encontrar os animes que possam lhe interessar. A proposta deste trabalho é criar uma aplicação com características da Web 2.0 para ajudar este público a  encontrar novos animes de acordo com seu gosto.
	\par
	\textbf{Palavras-chaves}: Wordpress, Web 2.0, Inteligência coletiva.
\end{resumo}

%=======================================================
% resumo em inglês
%=======================================================
%\begin{resumo}[Abstract]
% \begin{otherlanguage*}{english}
%   This is the english abstract.
%
%   \vspace{\onelineskip}
% 
%   \noindent 
%   \textbf{Key-words}: latex. abntex. text editoration.
% \end{otherlanguage*}
%\end{resumo}
% ---


%=================================================================================================
% Listas e Indices
%=================================================================================================

%=======================================================
% inserir lista de ilustrações
%=======================================================
%\pdfbookmark[0]{\listfigurename}{lof}
%\listoffigures*
%\cleardoublepage

%=======================================================
% inserir lista de tabelas
%=======================================================
%\pdfbookmark[0]{\listtablename}{lot}
%\listoftables*
%\cleardoublepage

%=======================================================
% inserir lista de abreviaturas e siglas
%=======================================================
%\begin{siglas}
%  \item[Fig.] Area of the $i^{th}$ component
%  \item[456] Isto é um número
%  \item[123] Isto é outro número
%  \item[lauro cesar] este é o meu nome
%\end{siglas}

%=======================================================
% inserir lista de símbolos
%=======================================================
%\begin{simbolos}
%  \item[$ \Gamma $] Letra grega Gama
%  \item[$ \Lambda $] Lambda
%  \item[$ \zeta $] Letra grega minúscula zeta
%  \item[$ \in $] Pertence
%\end{simbolos}

%=======================================================
% inserir o sumario
%=======================================================
\pdfbookmark[0]{\contentsname}{toc}
\tableofcontents*
\cleardoublepage


%=================================================================================================
% ELEMENTOS TEXTUAIS
%=================================================================================================
\textual

%=======================================================
% Introdução
%=======================================================
\chapter{Introdução}
\section{Descrição do problema}
Como é possível de observar o crescente volume de informações na Web 2.0 apresenta muitas vantagens, porém um problema cada vez mais difícil de se superar é encontrar o que você procura. Há muita informação que carece de um agrupamento melhor, assim como uma boa organização para que as pessoas possam encontrar. Neste trabalho meu objeto de estudos será uma comunidade de pessoas com um interesse em comum, animações japonesas chamadas de animes e revistas em quadrinhos que recebem o nome de mangá.
\par

Não só dentre os brasileiros, mas sim no mundo todo, a cultura oriental está ganhando um destaque, e a Web, é em um dos fatores responsáveis por isso tendo em vista que ela remove parte dos empecilhos geográficos para se conhecer uma cultura e também incentiva aos usuários produzirem conteúdo, interagirem e trocar informações sobre muitos assuntos.
\par

No Japão, há várias empresas de entretenimento dedicadas a produção de animes e mangás como o Studio Ghibli, a T\~oei Animation e Animax são exemplos de alguns deles. Essas animações chegam por serem consideradas tão diferentes das demais, chegam a ter uma classificação que difere bastante das classificações formais usadas para classificar histórias. Essas classificações são usadas para descrever o público alvo e parte da ambientação. Por exemplo, um anime ou mangá infantojuvenil é chamado de \textit{Sh\~onen}, que geralmente tem em seu contexto batalhas com grandes demonstrações de poder e tem como foco um público masculino jovem. Já \textit{Sh\~ojo} tem como público alvo meninas e sua temática geralmente envolve romance ou uma comédia romântica.
\par

Como é possível observar, nos exemplos citados, uma história infantojuvenil pode receber duas classificações diferentes, porem ainda há mais classificações quanto ao enredo. Mechas são animes de robôs gigantes, \textit{Mah\~o Sh\~onen} que tem como tema um garoto com poderes mágicos, são alguns dos exemplos. Esse grande volume de classificações podem ser combinadas para classificar um anime ou mangá. Porem por maior que seja essa lista, há ainda alguns elementos que não são englobados por essa classificação, fazendo com que dois animes com características que diferem possam ser classificados da mesma forma. Por exemplo caso seja um \textit{Sh\~onen} típico com muitas lutas, porem 1 deles é com "briga de rua" e outro é com "ninjas" esses animes podem ser muito diferentes, porem ambos podem receber apenas a classificação "\textit{Sh\~onen}", tendo em vista que não há classificação específica para "briga de rua" ou "ninjas".
\par

Como a classificação japonesa pode permitir que animes com temáticas diferentes possam ser classificados da mesma forma, uma pessoa poderia por exemplo gostar muito de animes \textit{Sh\~onen}, contudo não gostar de ninjas. Essa pessoa poderia por engano, ser indicada a um anime com muitos ninjas no enredo mesmo sem gostar dessa temática.
\par

Um hábito de muitas pessoas deste grupo é assistir vários animes e ler vários mangás ao mesmo tempo. Atitude essa que somada ao crescente número de mangás e animes disponíveis na Web, tende a fazer com que essas pessoas esqueçam em qual episódio estavam em determinado anime, muitas vezes vendo algum episódio mais a frente, que pode revelar algo do enredo, ou gastando muito tempo para encontrar qual o episódio que estava para poder prosseguir.
\par

Nos animes e mangás, há diversas obras que se referem a um mesmo mundo, enredo ou contexto como \textit{Gundam 0080: A War in Pocket}, \textit{Zeta Gundam}, \textit{Gundam ZZ} e  \textit{Char's CounterAttack} são alguns dos títulos fazem referencia a um mesmo mundo pertencentes a uma franquia chamada \textit{Mobile Suit Gundam} popularmente conhecida como  \textit{Gundam}. Essa mesma franquia possui diversos outros títulos e também muitas \textit{Side Stories}, que são histórias paralelas aos eventos principais como \textit{Mikkai Amuro to Lalah} e \textit{Animage Bunko}.
\par
Como é possível observar os nomes das histórias nem sempre tem uma conexão direta, embora todas elas sejam da mesma franquia, algumas delas sejam da linha principal de eventos. Isso serve para ilustrar uma possível dificuldade que o grupo de seguidores enfrentam ao tentar descobrir mais sobre esta franquia.
\par

Ainda há o problema de se acompanhar uma obra que está sendo lançada ou uma obra que já possui muitos episódios. Por exemplo, o anime  \textit{One Piece} teve seu primeiro episódio lançado no ano de 1997 ele conta com um anime com mais de 600 episódios e um mangá com mais de 750 capítulos, ambos contando a mesma história. Esse anime ainda está em lançamento e sem uma previsão para terminar sendo uma das histórias de maior sucesso de todos os tempos.
\par

O \textit{One Piece} é um exemplo que existem histórias muito longas, e qualquer pessoa que comece a assistir hoje pode ter dificuldade em se lembrar sempre de qual foi o ultimo episódio assistido, sofrendo assim um problema similar das pessoas que assistem a diversos animes e mangas simultâneos.
\par


\section{Objetivo geral}

A proposta deste trabalho, é estudar as funcionalidades das aplicações da Web 2.0 para desenvolver uma plataforma que ajude a solucionar todos ou ao menos alguns destes problemas.
\par

\subsection{Objetivos específicos}
\subsubsection{Replicável}
Essa plataforma também deve ser abrangente e versátil o suficiente para permitir que, uma vez pronta, a solução possa ser replicada em ambientes com problemas similaridades, tais como filmes, livros, seriados e HQ (histórias em quadrinhos) ou bibliotecas, agrupando os livros por assuntos.
\par

\subsubsection{Inteligência coletiva}
Como se trata de uma plataforma web, o conceito de web 2.0 estará presente na forma de inteligência coletiva. A plataforma proposta, deverá ser alimentada por fans, hoje há diversos sites que tem como objetivo fornecer animes, com legenda, para qualquer pessoa que tenha interesse. A um grande número desses sites, chamados de \textit{fansub} ou \textit{fansubs}, são sites mantidos em grande parte por admiradores que desejam espalhar a cultura japonesa e doações de usuários que se sentem compelidos a ajudar.
\par 

Esse esforço coletivo para manter um \textit{hobby}, será necessário para o sucesso da plataforma, pois ela deverá ser mantida pelos próprios usuários. Tal decisão é necessária pois há um grande volume de novos animes e mangas todos os dias e poucas pessoas podem ter muita dificuldade em manter o conteúdo atualizado e organizado.
\par

\subsubsection{Saber o que foi visto}
A plataforma deve, oferecer recursos para que seus usuários possam saber quais episódios de anime, capítulos de mangá ou equivalentes, já foram vistos. Esse recurso poderá ajudar  pessoas que acompanham diversos animes e mangas, pessoas que estão vendo um anime com muitos episódios e até mesmo pessoas que assistem animes esporadicamente se esquecem do nome dos animes já vistos.
\par

\subsubsection{Saber o irá ver antes de ver}
O problema de se escrever uma sinopse é que em muitos casos esta revela algum detalhe do enredo, algo que diversas pessoas não gostam. Por este motivo os animes devem ser classificados, tanto com a classificação japonesa quanto com \textit{tags}, para descrever elementos importantes na trama. No exemplo anteriormente citado de dois animes similares, ambos receberiam a classificação \textit{Sh\~onen}, porém poderiam ter \textit{tags} diferentes, como "Briga de rua" e "ninjas".
\par
Essas \textit{tags} tem como objetivo informar a temática ou algo relevante da história, porem sem revelar detalhes do enredo.
\par 

\subsubsection{Fornecer indicações}
A plataforma deve permitir que os usuários possam procurar novos animes com base na classificação ou nas  \textit{tags}. O objetivo dessa busca é que um usuário que tenha gostado de algum ou alguns elemento presente em um anime possa buscar animes com esses mesmo elementos, ou até mesmo criar um filtro misto, inserindo elementos que o agradou e excluindo animes com elementos que o desagrada.
\par
Tal funcionalidade irá ajudar os usuários a obterem indicações de animes que eles podem gostar mais, com base em seus interesses, ajudando assim que eles tenham mais facilidade de encontrar novos animes com um menor esforço.
\par

\subsection{Justificativa}
O número de pessoas que acompanham animes hoje em dia, já se tornou um número significativamente grande. Alguns poucos sites brasileiros,  como o site narutoPROJCT que possui conteúdo somente do anime de nome Naruto, possuem mais de 200 mil acessos diários e em dias que um novo episódio é lançado esse número dobra. Com tantas pessoas nessa comunidade que está crescendo cada dia mais e com diversos novos animes nos mais diversos assuntos e temáticas os problemas citados tendem a aumentar e a afetar um maior número de pessoas. Com tal número de pessoas, aderindo a um mesmo \textit{hobby}, é interessante resolver alguns dos problemas mais comuns deste meio, tendo em vista que é possível e interessante os novos mercados que podem surgir em torno dessa comunidade.
\par

%=======================================================
% TEXTO
%=======================================================
\chapter{Referencial Teórico}
%\section{Internet}
%\subsection{Primórdios}
%
%A internet pode ser considerada um produto vindo da guerra. Na busca desenfreada por uma vantagem durante a guerra fria os EUA queriam uma forma de ser ver a frente da URSS, e com isso buscavam sempre o desenvolvimento de novas tecnologias e inovações. \cite{historia-internt}
%\par
%
%Foi J.C.R. Licklider, do Instituto Tecnológico de Massachussets (MIT), que no ano de 1962 difundiu a idéia de  “rede galáctica”, que seriam todos os computadores da terra conectados com uma única forma de compartilhamento, idéia audaciosa, porem nos dias de hoje não é tão improvável assim. \cite{historia-internt}
%\par
%
%\subsubsection{ARPANET}
%
%A ARPANET é considerada por muitos a percursora da internet. Ela foi desenvolvida durante o período histórico conhecido como "Guerra Fria", pelos EUA que temendo um ataque soviético pudesse vir a perder tanto a comunicação quanto informações. \cite{ARPANET}
%\par
%
%A primeira mensagem foi enviada no ano de 1969, a ARPANET enviou a sua primeira mensagem com a palavra "login", que chegou incompleta por alguns erros, ao qual foram gastos alguns anos para a correção dos mesmos. \cite{historia-internt}
%\par
%
%\subsubsection{Década de 70}
%
%Muitas coisas foram feitas na década de 1970. Os cientistas Vinton Cerf e sua equipe, fizeram um experimento que era a conexão de três redes distintas, tal projeto era chamado de "interneting", termo que foi sendo vastamente usado e abreviado até receber o nome de Internet. \cite{historia-internt}
%\par
%
%Ainda no ano de 1970 Vinton Cerf, desenvolveu o protocolo TCP/IP que visava transmitir as mensagens sem erro, da origem ao destino. Porem tal protocolo só foi oficializado como único no ano de  1983.\cite{historia-internt}
%\par
%
%Ainda na mesma década, a primeira mensagem contendo um emoticon foi enviada, por Kevin MacKenzie que usou um símbolo para representar uma ironia. Em 1971, Bob Thomas, criou o que hoje conhecemos como vírus, que não fazia nada além de incomodar o usuário. \cite{historia-internt}
%\par
%
%Já no ano de 1973, a ARPANET foi ligada a Europa e a primeira conexão intercontinental foi estabelecida. Em 1977, nos EUA, já havia um número bem maior que os quatro primeiros servidores da ARPANET. Em 1979 Tom Truscott e Jim Ellis, interligaram computadores em uma rede de notícias divididos por categorias de interesse, o que fez com a rede não fosse somente interessante para maios científicos, mas sim para diversas pessoas.  \cite{historia-internt}
%\par
%
%
%\subsubsection{MILNET, ARPANET, Internet}
%
%Em 1983 com o crescimento da ARPANET e com o declínio da "Guerra Fria" a ARPANET, perdeu uma parte de seu caráter militar, e por este motivo, ela foi dividida em duas novas redes a MILNET que era militar e a ARPANET, que se tornou de uso civil. Em 1989 o segmento civil, foi desativado, porem serviu de para diversas redes interonectadas, o que hoje chamamos de Internet. \cite{historia-internt, web}
%\par
%
%\section{Evolução da Web}
%
%\subsection{Web}
%
%Em 1989, Tim Berners-Lee, criou a World Wide Web o que chamamos de WEB. Esse projeto tinha como objetivo ligar as universidades para compartilhar e usar mutuamente os trabalhos acadêmicos. Esse mesmo cientista criou o código HTML e o protocolo HTTP.  \cite{web}
%\par
%
%\subsubsection{Guerra dos navegadores}
%Em 1993, Marc Andreessen e Rob McCool inventaram o primeiro navegador chamado de Mosaic. Por dois anos, o Mosaic foi o principal meio para pessoas fora das universidades explorem a internet. O Objetivo dos criadores do programa era tornar o uso da internet algo prático para outras pessoas. Alguns anos depois  diversos navegadores foram surgindo, revolucionando o software com recursos novos e possibilidade de personalização. \cite{web}
%\par
%
%Em 1995 surge o Netscape, que foi o primeiro navegador comercial da internet, algum tempo depois a Microsoft lançou o seu navegador o Internet Explorer que passou a ser usado pela grande maioria dos usuários da rede. \cite{web}
%\par
%
%\subsection{Web 2.0}
%
%Como foi notado o objetivo da internet e da web sofreu muitas mudanças, passou de uma forma de manter e compartilhar informações militares, a uma rede que pode potencialmente conectar todos os computadores do mundo e partilhar informações, fazer amigos, descobrir coisas novas dentre diversas outras possibilidades. Neste momento pode-se perceber que a antiga Web, já havia extrapolado seu escopo inicial, e por isso havia se tornado algo novo, algo que chamamos de WEB 2.0. \cite{ARPANET, historia-internt, web, web2.0}
%\par
%
%A Web 2.0 não está também vinculada apenas a documentos científicos por uma rede. Ela mudou os objetivos com o passar do tempo. Na atualidade compartilhar arquivos acadêmicos entre universidades não pode mais ser chamado de prioridade. \cite{web2.0}
%\par
%
%Hoje o objetivo da chamada Web 2.0 vai muito além disso, temos hoje uma rede onde cada usuário tem o potencial de gerar informações, de participar para construir algo tanto de forma direta e cociente quanto de forma indireta e inconsciente. O Google se faz de exemplo da forma como vários usuários de forma inconsciente melhoram os serviços oferecidos pela empresa, com ferramentas como PageRank que é uma lista de popularidade muito bem trabalhada.\cite{google, web2.0}
%\par
%
%\subsubsection{Web 2.0 como comportamento}
%Podemos também observar que a proposta de \citeonline{web2.0} não há uma fronteira rígida que define WEB 2.0, mas sim uma espécie de "centro de características". Essa classificação leva a pensar que as aplicações tem características de WEB 2.0. Essa forma de ver a WEB 2.0 nos faz perceber que não é possível, ou até incorreto, tentar definir um limite solido para onde termina WEB 1.0 e onde começa WEB2.0.
%\par
%Contudo, ainda há uma forma de dizer se uma aplicação leva esse comportamento. \citeonline{web2.0} quando buscou definir web 2.0, ressaltou algumas características que as aplicações devem ter, para se adequarem melhor ao ambiente que ele definiu como sendo WEB 2.0. As características citadas por ele são:
%\par
%
%\begin{itemize}
%	\item{Emprego da inteligência coletiva.}
%	\item{Servços, não software “empacotado”.}
%	\item{Arquitetura de participação.}
%	\item{Escalabilidade de custo eficiente.} 
%	\item{Fonte e transformação de dados remixáveis.}
%	\item{Software em mais de um dispositivo.}
%	\item{Uso da folksonomy.}
%\end{itemize}


\section{Web 2.0}

\subsection{ Início da internet}
\par
A internet pode ser considerada um produto vindo da guerra. Na busca desenfreada por uma vantagem durante a guerra fria os EUA queriam uma forma de ser ver a frente da URSS, e com isso buscavam sempre o desenvolvimento de novas tecnologias e inovações. \cite{historia-internt}

\par
A ARPANET é considerada por muitos a percursora da internet. Ela foi desenvolvida durante o período histórico conhecido como "Guerra Fria", pelos EUA que temendo um ataque soviético pudesse vir a perder tanto a comunicação quanto informações. \cite{historia-internt}
\par

Em 1989, Tim Berners-Lee, criou a World Wide Web o que chamamos de WEB. Esse projeto tinha como objetivo ligar as universidades para compartilhar e usar mutuamente os trabalhos acadêmicos. Esse mesmo cientista criou o código HTML e o protocolo HTTP.  \cite{web}
\par

\subsection{Web 2.0}

Como pode ser observado os objetivos da internet e da web sofreram muitas mudanças, tendo em vista que passaram de uma forma de manter e compartilhar informações militares, a uma rede que pode potencialmente conectar todos os computadores do mundo, partilhar informações, fazer amigos, descobrir coisas novas dentre diversas outras possibilidades. Neste momento pode-se perceber que a antiga Web, já havia extrapolado seu escopo inicial, e por isso havia se tornado algo novo, algo que chamamos de WEB 2.0. \cite{historia-internt, web, web2.0}
\par

A Web, tinha um caráter menos dinâmico, e interativo. As pessoas apenas buscavam informações de seus campos de interesse, e produzir conteúdo, informação, partilhar opiniões conhecer pessoas, era algo mais difícil. A internet era de certa forma alimentada por uma via de mão única, pois os sites e aplicativos eram feitos para funcionar de uma forma menos interativa. \cite{web2.0}
\par

Com o passar dos anos, algumas aplicações ganharam funcionalidades e comportamentos novos. Começaram a permitir a interação dos usuários, transformando cada usuário em um possível gerador de conteúdo. Ocorreu a criação de Blogs e Wikis.\cite{web2.0}

Foi possível observar a evolução de diversas aplicações, e percebeu-se que haviam muitas características diferentes, porem todas elas eram convergem no objetivo de transformar a Web em um ambiente onde a força da inteligência coletiva era aplicada.\cite{web2.0}

\citeonline{web2.0} também mostra que não é interessante, ou até mesmo fácil, colocar uma barreira rígida que dita os limites entre Web e Web 2.0. Ele mostra que há um centro de características, que remete a idéia que de quanto mais daquelas características uma aplicação mostra, mais ela está próxima dessa nova essência. 
\par

Contudo, limitar a definição de Web 2.0 a "um grupo" de características seria uma definição muito pobre e que rapidamente poderia ser superada. \citeonline{web2.0} fez mais que definir como características, ele transmite a idéia que Web 2.0 é uma forma como as aplicações devem se comportar, ele transmite a idéia que uma aplicação com "cara de Web 2.0" deve ser uma aplicação inteligente ao ponto de fazer uso da inteligência coletiva, permitir que os usuários se tornem produtores de conteúdo juntamente com consumidores.

Um "site cm cara de Web 2.0" deve usar a internet como plataforma, e fazer uso inteligente de seus usuários, tornando-se melhor cada vez que aumenta o número de usuários. Uma aplicação inteligente e comunitária onde todas as pessoas que participam sejam potenciais construtores.  \cite{web2.0}

\section{Características}
Dentre diversas características da Web 2.0 algumas são mais relevantes no contexto deste trabalho e elas serão melhores descritas a seguir. Porem é importante informar que essas não são as únicas características de uma aplicação Web 2.0 pois ainda há algumas outras tais como arquitetura de participação, software em mais de um dispositivo e escalabilidade de custo eficiente que estão presentes em diversas aplicações Web mas ,neste trabalho, são de menor importância. \cite{web2.0}

\subsection{Inteligência coletiva}

No, \citeonline{inteligencia-coletiva}, é traçada a idéia de inteligência coletiva. O autor a define como uma forma de diversos indivíduos contribuírem com seus conhecimentos para juntos obterem o todo. Baseando-se na crença de que apesar de "nenhuma" pessoa ter o conhecimento do "todo", esse mesmo "todo" pode ser descoberto a partir de fragmentos individuais, contidos em cada indivíduo. Essa idéia tem muitas implicações sociais, tais como todo indivíduo tem algo a acrescentar e nenhuma pessoa é desprovida de conhecimento.
\par
Antes de se valorizar o conhecimento, essa abordagem valoriza as pessoas como indivíduos, capazes de produzir um conhecimento. Essa abordagem também sugere que todas as pessoas são capazes de produzir conhecimento relevante, por isso é uma abordagem que valoriza o homem antes do conhecimento. Tal abordagem será muito útil quando observarmos a inteligência coletiva na web 2.0. \cite{inteligencia-coletiva}
\par

O conceito de inteligência coletiva é tão importante que \citeonline{inteligencia-coletiva-2}, continua com a definição de \citeonline{inteligencia-coletiva}. Ele diz que para se fazer um uso eficiente da inteligência coletiva, não deve existir apenas indivíduos, mas sim "times" com um objetivo comum e uma estratégia. Esse complemento a definição de \citeonline{inteligencia-coletiva} é uma muito interessante, porem em alguns exemplos citados por, \citeonline{web2.0}, em empresas como Google, podemos identificar o uso eficiente da inteligência coletiva, sem que os indivíduos estejam trabalhando como um time, ou sequer conscientemente com um foco comum.
\par

A inteligência coletiva é um fator tão forte e importante no conceito de WEB 2.0 que \citeonline{web2.0} chega a afirmar que é uma parte essencial da WEB 2.0. Essa alegação pode não ser totalmente verdade, porem ainda assim, é possível perceber o quão importante é o uso da inteligência coletiva. Isso mostra que Web 2.0 tende a valorizar os indivíduos de forma singular, respeitando o conhecimento deles.
\par

Esse fenômeno de criação de conteúdo coloca cada vez mais na mão dos indivíduos o controle da informação e do conhecimento. Como na web 2.0 a informação é gerada por usuários, muita informação é criada a todo instante. \citeonline{web2.0} chega a citar a teoria \textit{"Long tail"} ou "cauda longa" para dizer que na web 2.0 grande parte do conteúdo é proveniente de pequenos sies ou blogs, ele também chega a dicar uma parte significativa de seu texto para falar sobre blogs e wiki, mostrando novamente o poder da inteligência coletiva.
\par

\subsection{“Folksonomia”}


Como foi observado por \citeonline{web2.0}, o crescente número de blogs, wikis e outra formas dos usuários gerarem conteúdo, também foi observado que os próprios usuários começaram a classificar seus conteúdos. Porem diferente da taxinomia usada em ambientes mais formais, como bibliotecas, essa classificação era mais popular e tentava passar um significado mais próximo da associação feita pelo cérebro das pessoas, e não por uma classificação rígida.
\par

\citeonline{web2.0} chama essa classificação mais informal, porem extremamente usada,de \textit{"folksonomy"}, ela foi concebida da junção das palavras \textit{"folks"} (Povo, pessoal, pessoas) com a junção da palavra \textit{"taxonomy"}(nome dado ao processo de classificação de palavras-chave), tentando expressar a idéia de "Classificação do povo".\cite{tag-2}
\par 

Essa nova classificação, é chamada de \textit{"tag"} ou etiqueta. Essas \textit{tags} cresceram de forma tão estrondosa que hoje são usadas para classificar imensas quantidades de conteúdo web. Hoje já se traça um "estudo" para saber como melhor classificar seus conteúdos, para que essas palavras-chave ajudem mecanismos de busca a encontrar conteúdo realmente significativo.\cite{web2.0, tag, tag-2}
\par

Essa classificação por palavras-chave, por contar com a força de um número potencialmente muito grande de pessoas, tende a representar melhor o conteúdo, ou parte dele, para outras pessoas. Essa idéia é apresentada de forma superficial em \citeonline{tag}, porem desenvolvida de forma clara pela autora \citeonline{tag-2}.
\par
Em seu artigo ela explica algumas dificuldades em utilizar o sistema tradicional de indexação, e cita alguns exemplos e artigos em que a forma tradicional dificulta o trabalho em algumas pesquisas, afirmando inclusive que pesquisadores em muitos casos tem que reiniciar sua busca diversas vezes, pois não há uma ligação hierárquica dentre os assuntos em questão.\cite{tag-2}
\par

Embora ela use de alguns exemplos de pesquisadores acadêmicos, ela não se atem a tal situação. Ela também explica que devido ao grande, e crescente, número de usuários, a web 2.0 gera uma quantidade gigantesca de informações, essas informações tem um problema para serem agrupadas de forma correta, pois usuários podem colocar \textit{tags} diferentes, e talvez até mesmo com significados distantes, por não conhecer determinado assunto.\cite{tag-2}
\par

Embora haja este problema, \citeonline{tag-2} também exemplifica com plataformas que resolvem isso colocando as \textit{tags} para serem discutidas entre os usuários. Tal atitude faz com que as \textit{tags} sejam mais precisas e mais elaboradas. A idéia de fazer com que os usuários discutam as \textit{tags} é para reduzir a complexidade das buscas, e fazer com que seja possível encontrar o seu objeto de interesse.\cite{tag-2}


%=======================================================
% Finaliza a parte no bookmark do PDF, para que se inicie o bookmark na raiz
%=======================================================
\bookmarksetup{startatroot}% 


%=======================================================
% Conclusão
%=======================================================
%\chapter*[Conclusão]{Conclusão}
%\addcontentsline{toc}{chapter}{Conclusão}


%=================================================================================================
% ELEMENTOS PÓS-TEXTUAIS
%=================================================================================================
\postextual


%=======================================================
% Referências bibliográficas
%=======================================================
\bibliography{Referencial}

%=======================================================
% Glossário
%=======================================================
% Consulte o manual da classe abntex2 para orientações sobre o glossário.
%\glossary

%=================================================================================================
% Apêndices
%=================================================================================================

%=======================================================
% Inicia os apêndices
%=======================================================
\begin{apendicesenv}
\partapendices

\chapter{Cronograma para TCC II}

\begin{center}
	\begin{tabularx}{\textwidth}{|X|X|X|X|X|}
		\hline
		
		Data de início & Fase & Objetivo & Resultado esperado & Data final \\ \hline
		
		28/07/2014 & Definição e descrição da metodologia & Escolher e definir formalmente a metodologia que será adotada & Escolha, definição e descrição da uma metodologia de desenvolvimento para o trabalho & 08/08/2014 \\ \hline
		
		08/08/2014 & Administrativo & Completar o administrativo & Completar a área restrita que permite adicionar animes e todas as informações diretamente ligadas & 31/08/2014 \\ \hline
		
		01/09/2014 & Plataforma básica & Frente do sistema & Criar todas as páginas necessárias para visualizar os animes & 30/09/2014 \\ \hline

		01/10/2014 & Funcionalidades & Funcionalidades da aplicação & Integrar com sistema as funcionalidades avançadas que foram planejadas & 30/10/2014 \\ \hline

		01/11/2014 & Revisão & Revisão geral & Revisar e melhorar todos os artefatos da monografia & 20/11/2014 \\ \hline
		
		21/11/2014 & Apresentação &  &  & 20/11/2014 \\ \hline


%		21/11/2014 & Apresentação &  &  & 31/11/2014 \\ \hline

	\end{tabularx}
\end{center}


\end{apendicesenv}


%=================================================================================================
% Anexos
%=================================================================================================

%=======================================================
% Inicia os anexos
%=======================================================
%\begin{anexosenv}
%
% Imprime uma página indicando o início dos anexos
%\partanexos
%
% ---
%\chapter{Morbi ultrices rutrum lorem.}
% ---
%\lipsum[30]
%
% ---
%\chapter{Cras non urna sed feugiat cum sociis natoque penatibus et magnis dis
%parturient montes nascetur ridiculus mus}
% ---
%
%\lipsum[31]
%
% ---
%\chapter{Fusce facilisis lacinia dui}
% ---
%
%\lipsum[32]
%
%\end{anexosenv}


%=======================================================
% INDICE REMISSIVO
%=======================================================

\printindex

\end{document}







